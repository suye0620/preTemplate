\usepackage[UTF8, heading = false, scheme = plain]{ctex}



\newfontfamily\WRYaHei{微软雅黑}%新建微软雅黑字体,注意,使用\newfontfamily命令创建的字体只能用于英文。是吗?是的,fontspec宏包提供了\fontspec、\setmainfont、\setsansfont、\setmonofont、\newfontfamily命令,当使用ctex宏包的时候,这些命令只对英文和阿拉伯数字有效。而ctex所使用的xeCJK宏包里所提供的\setCJKmainfont、\setCJKsansfont、\setCJKmonofont、\setCJKmonofont、\setCJKmonofont、\setCJKfamilyfont、\setCJKfallbackfamilyfont则只对CJK字体有效。
\setCJKfamilyfont{WRYaHei}{微软雅黑}%新建微软雅黑字体,注意,使用\setCJKfamilyfont命令创建的字体只能用于中文。
\newcommand{\WRYaHeiZH}{\WRYaHei\CJKfamily{WRYaHei}}%新建一个命令,对中文和英文都使用微软雅黑
\setCJKmainfont{微软雅黑}
\setsansfont{Times New Roman}%这个是用来设置正文中的英语使用Times New Roman的,(可是为什么是这一个而不是另外的两个呢?)

%常规大小
\renewcommand{\normalsize}{\fontsize{12}{12}\selectfont}%把常规大小设置为12pt
%标题字体
\setbeamerfont{title}{family=\WRYaHeiZH,size=\fontsize{28}{28}\selectfont,series=\bfseries}%设置标题的字体为微软雅黑,加粗,字号为28pt
\setbeamerfont{subtitle}{family=\WRYaHeiZH,size=\fontsize{14}{14}\selectfont,series=\bfseries}%设置标题的字体为微软雅黑,加粗,字号为28pt


%机构字体
\setbeamerfont{institute}{family=\songti,size=\fontsize{12}{12}\selectfont,series=\bfseries}%设置机构的字体为宋体,加粗,字号为12pt
%作者字体
\setbeamerfont{author}{parent=institute}%设置作者的字体和机构的字体相同
%日期字体
\setbeamerfont{date}{parent=institute}%设置日期的字体和机构的字体相同
%list元素的字体
\setbeamerfont{item projected}{size=\fontsize{12}{12}\selectfont,series=\mdseries}
\setbeamerfont{itemize/enumerate subbody}{family=\kaishu,size=\fontsize{12}{12}\selectfont,series=\mdseries}

%帧标题字体
\setbeamerfont{frametitle}{family=\WRYaHeiZH,size=\fontsize{18}{18}\selectfont,series=\bfseries}%设置帧标题的字体为微软雅黑,加粗,字号为18pt
%帧子标题字体
\setbeamerfont{framesubtitle}{family=\kaishu,size=\fontsize{14}{14}\selectfont,series=\bfseries}%设置帧标题的字体为微软雅黑,加粗,字号为14pt

%页眉字体
%\setbeamerfont{headline}{size=\fontsize{12}{12}\selectfont}%设置页眉的字号为16pt,没有办法呀,如果按照经验丰富的老专家的要求使用24pt的话,这里还是太丑了,我最大能接受的就是16pt了。如果这不行,那么只能不要页眉了

%页脚字体
\setbeamerfont{footline}{size=\fontsize{8}{8}\selectfont}%设置页脚的字体等于页眉字体
%盒子环境
%标题字体
\setbeamerfont{block title}{size=\fontsize{14}{14}\selectfont,series=\bfseries}%设置盒子环境的标题的字体为微软雅黑,加粗,字号为14pt


%%%%%%%%%%%%%%%%%%%%%%%%%%%%% 题注  %%%%%%%%%%%%%%%%%%%%%%%%%%%
\setbeamertemplate{caption}[numbered]
%\setbeamerfont{caption}{size=\scriptsize}
\setbeamerfont{caption}{size=\fontsize{10}{10}\selectfont}
\renewcommand\figurename{图}
\renewcommand\tablename{表}
%%%%%%%%%%%%%%%%%%%%%%%%%%%%%%%%%%%%%%%%%%%%%%%%%%%%%%%%%%%%%%%%



%%%%%%%%%%%%%%%%%%%%%%%%%%%    footline     %%%%%%%%%%%%%%%%%%%%%%%%%%%%%
\defbeamertemplate{footline}{NGEGFootlineTemplate}{%
	\leavevmode% 离开vmode,也就是离开竖直模式,进入水平模式
	\hbox{%
%		\begin{beamercolorbox}[wd=.3\paperwidth,ht=2.25ex,dp=1ex,center]{author in head/foot}%
%			\ifnum \the\value{page}>1 \usebeamerfont{author in head/foot}\insertshortauthor\fi
%		\end{beamercolorbox}%
%		\begin{beamercolorbox}[wd=.4\paperwidth,ht=2.25ex,dp=1ex,center]{title in head/foot}%
%			\ifnum \the\value{page}>1 \usebeamerfont{title in head/foot}\insertshorttitle\fi
%		\end{beamercolorbox}%
%		\begin{beamercolorbox}[wd=0.3\paperwidth,ht=2.25ex,dp=1ex,center]{date in head/foot}%
%			\ifnum \the\value{page}>1 \insertframenumber{} / \inserttotalframenumber\fi
%	\end{beamercolorbox}}%
%	\vskip0pt%
	\begin{beamercolorbox}[wd=.25\paperwidth,ht=2.25ex,dp=1ex,center]{author in head/foot}%
    \usebeamerfont{author in head/foot}\insertshortauthor
  	\end{beamercolorbox}%
  	\begin{beamercolorbox}[wd=.25\paperwidth,ht=2.25ex,dp=1ex,center]{title in head/foot}%
    \usebeamerfont{author in head/foot}\insertshortdate
  	\end{beamercolorbox}%
  	\begin{beamercolorbox}[wd=.4\paperwidth,ht=2.25ex,dp=1ex,center]{author in head/foot}%
    \usebeamerfont{title in head/foot}\insertshorttitle
  	\end{beamercolorbox}%
  	\begin{beamercolorbox}[wd=0.1\paperwidth,ht=2.25ex,dp=1ex,right]{title in head/foot}%
    \insertframenumber{} / \inserttotalframenumber\hspace*{2ex}
  	\end{beamercolorbox}}%
  	\vskip0pt%
}

\setbeamertemplate{footline}[NGEGFootlineTemplate]
%%%%%%%%%%%%%%%%%%%%%%%%%%%%%%%%%%%%%%%%%%%%%%%%%%%%%%%%%%%%%%%%%%%%%%%%%%%%%%%%%

% \RequirePackage{bbding}
% %\RequirePackage{arev}
% \newlength{\HeightOfItem}
% \newlength{\HeightOfSubItem}
% \settoheight{\HeightOfItem}{\usebeamerfont*{itemize/enumerate body}{item元素字体高度}}
% \settoheight{\HeightOfSubItem}{\usebeamerfont*{itemize/enumerate subbody}{subitem元素字体高度}}
% \setbeamercolor{structure}{fg=red,bg=white!90!gray}
% \defbeamertemplate{itemize item}{NGEGItemizeItemTemplate}
% 	{
% 		\raisebox{-0.2\HeightOfItem}{\PencilRightUp}
% 	}
% \setbeamertemplate{itemize item}[NGEGItemizeItemTemplate]
% 
% \defbeamertemplate{itemize subitem}{NGEGItemizeSubitemTemplate}
% 	{
% 		\raisebox{-0.2\HeightOfSubItem}{\HandRight}
% 	}
% \setbeamertemplate{itemize subitem}[NGEGItemizeSubitemTemplate]
% 
% 
% \defbeamertemplate{enumerate item}{NGEGEnumerateItemTemplate}
% 	{
% 		\usebeamercolor[structure]{item projected}
% 		\raisebox{-0.5\HeightOfItem}{\checkmark}
% 	}
% \setbeamertemplate{enumerate item}[NGEGEnumerateItemTemplate]

% 无序列表形状
\setbeamertemplate{itemize items}[triangle]
% 有序列表
\setbeamertemplate{enumerate items}[default]

\setbeamercolor{itemize item}{fg=red}
% 可以肯定的是,目录中的标题编号也使用了enumerate环境。故下面两个fg必须保持一致,不然修改不成功。
\setbeamercolor{enumerate item}{fg=red}
\setbeamercolor{section number projected}{fg=red,bg=white!90!gray}